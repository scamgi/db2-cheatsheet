% sections/jpa.tex

\section*{JPA (Java Persistence API)}

\subsection*{Part 1: The Core Workflow}
\begin{enumerate}
    \item \textbf{Read the text}: Analyze requirements, E-R diagram, cardinalities, and application access patterns.
    \item \textbf{Create basic entities}: For each entity in the E-R diagram, create a class annotated with \texttt{@Entity}.
    \item \textbf{Map primary keys and attributes}: Use \texttt{@Id} and \texttt{@GeneratedValue}.
    \begin{lstlisting}
@Entity
public class Building {
    @Id 
    @GeneratedValue(strategy=GenerationType.AUTO)
    private int buildingId;
    private String address;
    // ...
}
    \end{lstlisting}
    \item \textbf{Map relationships} ($\implies$ See Part 2).
    \item \textbf{Define queries and ordering}: Use annotations to define queries and specify data retrieval order.
    \begin{lstlisting}
// To define a reusable query
@NamedQuery(name="User.checkCredentials", 
    query="SELECT u FROM User u WHERE u.username = ?1 AND u.password = ?2")

// To specify date/time precision
@Temporal(TemporalType.TIMESTAMP)
private Date timestamp;

// To order a collection fetched from the database
@OneToMany(mappedBy="manager")
@OrderBy("end DESC")
private List<Project> managedProjects;
    \end{lstlisting}
    \item \textbf{Implement helper methods}: For bidirectional relationships, create methods to synchronize both sides.
\end{enumerate}

\section*{Part 2: Relationship Mapping}
\begin{center}
\begin{tabular}{|l|l|}
\hline
\textbf{Owner Side} & \textbf{Inverse Side (Non-Owner)} \\
\hline
Manages the relationship. & Delegates management to the owner. \\
Uses \texttt{@JoinColumn} to name the FK. & Uses \texttt{mappedBy} attribute. \\
\hline
\texttt{@ManyToOne}, \texttt{@OneToOne} & \texttt{@OneToMany}, \texttt{@OneToOne} \\
\hline
\multicolumn{2}{|c|}{\texttt{fetch = FetchType.LAZY} (for collections) or \texttt{FetchType.EAGER} (for single objects)} \\
\hline
\texttt{cascade = \{CascadeType.PERSIST, REMOVE, ...\}} & No cascade settings. \\
\texttt{orphanRemoval = true} & \\
\hline
\end{tabular}
\end{center}

\subsubsection*{One-to-Many Example (Building $\leftrightarrow$ Meter)}
This is the most common bidirectional relationship. The "Many" side is the owner.
\begin{lstlisting}
// Building.java (Inverse Side)
@OneToMany(mappedBy = "building", fetch = FetchType.EAGER)
private List<Meter> meters;

// Meter.java (Owner Side)
@ManyToOne
@JoinColumn(name = "buildFK")
private Building building;
\end{lstlisting}

\subsubsection*{Many-to-Many Example (Project $\leftrightarrow$ Employee)}
A join table is used, so either side can be the owner.
\begin{lstlisting}
// Project.java (Owner Side)
@ManyToMany(fetch = FetchType.EAGER)
@JoinTable(
    name="emp_prj",
    joinColumns={@JoinColumn(name="projid")},
    inverseJoinColumns={@JoinColumn(name="empid")}
)
private List<Employee> employees;

// Employee.java (Inverse Side)
@ManyToMany(mappedBy = "employees", fetch = FetchType.EAGER)
private List<Project> assignedProjects;
\end{lstlisting}

\section*{Part 3: Advanced Patterns (N:M with Attributes)}
When a Many-to-Many relationship has its own attributes (like `quantity` in an order), you have two main options.

\subsection*{\texttt{@ElementCollection}}
Use when the relationship attribute is simple (e.g., a number, a string). The collection becomes a \texttt{Map<Entity, AttributeType>}.
\begin{lstlisting}
// In Order.java
@ElementCollection(fetch = FetchType.EAGER)
@CollectionTable(name = "product_order", 
    joinColumns = @JoinColumn(name = "orderId"))
@MapKeyJoinColumn(name = "productId") // FK for the Map Key (Product)
@Column(name = "quantity")            // Column for the Map Value (Integer)
private Map<Product, Integer> products;
\end{lstlisting}

\subsection*{Association Entity}
Objectify the relationship into its own entity. Use this when the relationship is complex or needs to be referenced directly.
\begin{enumerate}
    \item \textbf{Create a Composite Key Class} (\texttt{@Embeddable})
    \begin{lstlisting}
@Embeddable
public class OrderLineId implements Serializable {
    private Long orderId;
    private Long productId;
    // ...constructors, getters, equals, hashCode
}
    \end{lstlisting}
    \item \textbf{Create the Association Entity} (\texttt{@Entity})
    \begin{lstlisting}
@Entity
public class OrderLine {
    @EmbeddedId
    private OrderLineId id;

    @ManyToOne
    @MapsId("orderId") // Maps to the 'orderId' field in OrderLineId
    private Order order;

    @ManyToOne
    @MapsId("productId") // Maps to the 'productId' field in OrderLineId
    private Product product;

    private int quantity;
}
    \end{lstlisting}
\end{enumerate}

\section*{Part 4: Operation-Cascading Cheat Sheet}
\begin{center}
\begin{tabular}{|l|p{4cm}|p{4cm}|p{4cm}|}
\hline
\textbf{Directive} & \texttt{cascade = CascadeType.PERSIST} & \texttt{cascade = CascadeType.REMOVE} & \texttt{orphanRemoval = true} \\
\hline
\textbf{Trigger} & The parent entity is persisted (e.g., \texttt{em.persist(parent)}). & The parent entity is removed (e.g., \texttt{em.remove(parent)}). & The parent-child link is broken. This happens if: \newline - The parent is removed. \newline - The child reference on the parent is set to \texttt{null}. \newline - The child is removed from the parent's collection. \\
\hline
\textbf{Effect} & The associated child entity is also persisted. & The associated child entity is also removed from the database. & The "orphaned" child entity is removed from the database. \\
\hline
\textbf{Use Case} & Automatically save a new \texttt{Address} when saving a new \texttt{Employee}. & Automatically delete all of a building's \texttt{Meters} when the \texttt{Building} is deleted. & Automatically delete an \texttt{Expense} report if it is unlinked from its \texttt{Mission}, ensuring no expense report exists on its own. \\
\hline
\textbf{Key Point} & Manages the \textbf{creation} lifecycle. & Manages the \textbf{deletion} lifecycle based on the parent's deletion. & Manages the deletion lifecycle based on the \textbf{relationship status}. Stronger than \texttt{CascadeType.REMOVE}. \\
\hline
\end{tabular}
\end{center}