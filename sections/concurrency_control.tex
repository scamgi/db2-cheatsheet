% sections/concurrency_control.tex

\section*{Concurrency Control}

\subsection*{Anomalies}
\begin{itemize}
    \item \textbf{Dirty Read:} Reading uncommitted data.
    \item \textbf{Lost Update:} One transaction's update is overwritten by another.
    \item \textbf{...:} ...
\end{itemize}

\subsection*{Protocols \& Classes}
[INSERT HERE the image about the classes]
\begin{itemize}
    \item \textbf{VSR (View-Serializable):} Schedules are equivalent to some serial schedule from a view perspective.
    \item \textbf{CSR (Conflict-Serializable):} Schedules can be transformed into a serial schedule by swapping non-conflicting operations. Precedence graph must be acyclic.
    \item \textbf{2PL (Two-Phase Locking):} A transaction has a growing phase (acquiring locks) and a shrinking phase (releasing locks). Cannot acquire after releasing.
    \item \textbf{Strict-2PL:} All exclusive (write) locks are held until the transaction commits or aborts. Prevents dirty reads.
    \item \textbf{Obermarck's Algorithm:} Used for distributed deadlock detection.
    \item \textbf{Timestamp Ordering (TS-Mono):} Uses timestamps to order transactions. Monotonic clock.
    \item \textbf{Timestamp Ordering (TS-Multi):} Multi-version concurrency control (MVCC).
    \item \textbf{Hierarchy Locking:} A protocol for locking objects at different granularities (e.g., table, page, row).
\end{itemize}

\subsection*{How to Determine if a Schedule is CSR}
To determine if a schedule is Conflict-Serializable (CSR), you must check if its precedence (or conflict) graph is acyclic. If there are no cycles, the schedule is CSR.

\begin{enumerate}
    \item \textbf{Identify Conflicts:} Find all conflicting operations. Two operations from different transactions conflict if they access the same data item and at least one is a write.
          \begin{itemize}
              \item Read-Write
              \item Write-Read
              \item Write-Write
          \end{itemize}
    \item \textbf{Draw the Precedence Graph:}
          \begin{itemize}
              \item \textbf{Nodes:} Create one node for each transaction in the schedule.
              \item \textbf{Edges:} Draw a directed edge from transaction \textbf{$T_i$} to \textbf{$T_j$} if an operation in $T_i$ conflicts with and occurs \textit{before} an operation in $T_j$.
          \end{itemize}
    \item \textbf{Check for Cycles:}
          \begin{itemize}
              \item If the graph has \textbf{no cycles}, the schedule is \textbf{CSR}. Example:

                    \begin{verbatim}
s: w_3
u: w_5 w_2 w_1
x: r_1 r_5 w_3
y: r_2 w_3 r_4
z: w_5 r_5
\end{verbatim}


                    \begin{tikzpicture}[
                            node_style/.style={
                                    circle,
                                    draw=black,
                                    fill=cyan!30,
                                    minimum size=1cm,
                                    font=\Large
                                },
                            arrow_style/.style={
                                    -Stealth,
                                    very thick
                                },
                            gray_arrow_style/.style={
                                    -Stealth,
                                    thick,
                                    gray
                                }
                        ]

                        % Define nodes
                        \node[node_style] (n1) at (1, 2) {1};
                        \node[node_style] (n2) at (-2, 0) {2};
                        \node[node_style] (n3) at (3, 0) {3};
                        \node[node_style] (n4) at (4.5, 2) {4};
                        \node[node_style] (n5) at (0, -2) {5};

                        % Draw edges
                        \draw[arrow_style] (n1) to (n3);
                        \draw[arrow_style] (n2) to (n1);
                        \draw[arrow_style] (n3) to (n4);
                        \draw[arrow_style] (n5) to (n2);
                        \draw[gray_arrow_style] (n5) to (n1);

                    \end{tikzpicture}


              \item If the graph has \textbf{one or more cycles}, the schedule is \textbf{not CSR}. Example:

                    \begin{Verbatim}[commandchars=\\\{\}]
s: \textcolor{red}{\textbf{w_2}} w_3
u: \textcolor{red}{\textbf{r_2}} w_5 w_2 w_1
x: r_1 r_5 w_3
y: r_2 w_3 r_4
z: w_5 r_5
                    \end{Verbatim}

                    \begin{tikzpicture}[
                            node_style/.style={
                                    circle,
                                    draw=black,
                                    fill=cyan!30,
                                    minimum size=1cm,
                                    font=\Large
                                },
                            arrow_style/.style={
                                    -Stealth,
                                    very thick
                                },
                            red_arrow_style/.style={
                                    -Stealth,
                                    very thick,
                                    red,
                                    bend right=45
                                }
                        ]

                        % Define nodes
                        \node[node_style] (n1) at (1.5, 2) {1};
                        \node[node_style] (n2) at (-1.5, 0) {2};
                        \node[node_style] (n3) at (3.5, 0) {3};
                        \node[node_style] (n4) at (5, 2) {4};
                        \node[node_style] (n5) at (0, -2) {5};

                        % Draw edges
                        \draw[arrow_style] (n1) to (n3);
                        \draw[arrow_style] (n2) to (n1);
                        \draw[arrow_style] (n3) to (n4);
                        \draw[arrow_style] (n5) to (n1);
                        \draw[arrow_style] (n5) to (n2);
                        \draw[red_arrow_style] (n2) to (n5);

                    \end{tikzpicture}

          \end{itemize}
\end{enumerate}