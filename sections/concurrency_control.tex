% sections/concurrency_control.tex

\section*{Concurrency Control}

\subsection*{Anomalies}
\begin{itemize}
    \item \textbf{Dirty Read:} Reading uncommitted data.
    \item \textbf{Lost Update:} One transaction's update is overwritten by another.
    \item \textbf{...:} ...
\end{itemize}

\subsection*{Protocols \& Classes}
[INSERT HERE the image about the classes]
\begin{itemize}
    \item \textbf{VSR (View-Serializable):}
    \begin{itemize}
        \item If the schedule is CSR, then it is VSR.
    \end{itemize}
    \item \textbf{CSR (Conflict-Serializable):} Schedules can be transformed into a serial schedule by swapping non-conflicting operations. Precedence graph must be acyclic.
    \item \textbf{2PL (Two-Phase Locking):} A transaction has a growing phase (acquiring locks) and a shrinking phase (releasing locks). Cannot acquire after releasing.
    \item \textbf{Strict-2PL:} All exclusive (write) locks are held until the transaction commits or aborts. Prevents dirty reads.
    \item \textbf{Obermarck's Algorithm:} Used for distributed deadlock detection.
    \item \textbf{Timestamp Ordering (TS-Mono):} Uses timestamps to order transactions. Monotonic clock.
    \item \textbf{Timestamp Ordering (TS-Multi):} Multi-version concurrency control (MVCC).
    \item \textbf{Hierarchy Locking:} A protocol for locking objects at different granularities (e.g., table, page, row).
\end{itemize}
